Durch das Mining, welches bei Kryptow\"ahrungen, die auf Proof of Work setzen zum Einsatz kommt, wurden im Jahre 2017 alleine bei Bitcoin 29 TWh\footnotemark[1] verbraucht.
Neue Ans\"atze versprechen die Gew\"ahrleistung der Funktionsweise ohne Mining und dem damit verbundenen Stromverbrauch.
Durch die Abkehr vom Mining als Grundbaustein f\"ur die Sicherheit von Kryptow\"ahrungen ergeben sich neue Chancen, aber auch neue Herausforderungen, beim Design und Risiken im Einsatz.
Ziel dieser Arbeit ist die Vorstellung einer Alternative zu Proof of Work, welche korrektes Verhalten der Protokollteilnehmer durch ein pfand\"ahnliches Konzept motiviert sowie vereinzelte Gegen\"uberstellungen des neuen, Proof of Stake genannten, Verfahrens mit Proof of Work.
Des Weiteren die Vorstellung von Angriffen in diesem Umfeld, Mechanismen zur Verhinderung und, wo möglich, das Ziehen von Vergleichen zu Proof of Work.
