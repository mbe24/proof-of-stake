Dieser Abschnitt befasst sich mit der Definition einiger Begriffe, die für das Verständnis dieser Arbeit wichtig sind. Weitere Begriffe werden bei der Verwendung definiert oder ergeben sich aus dem Kontext. 

\subsection{Staking}

\textit{Stake} bezeichnet das Pfand, welches \textit{Validatoren} bezahlen.
Kennzeichnend für dieses Stake ist, dass es in der Kryptowährung des Systems, das durch das Proof-of-Stake-Verfahren gesichert werden soll, bezahlt wird \cite[S. 1]{casper_ffg}. Dieser Vorgang kann als \textit{Staking} bezeichnet werden. Es lässt sich ein Henne-Ei-Problem erkennen.

Das \textit{Bootstrapping} von Proof-of-Stake-Systemen ist im Gegensatz zu Proof-of-Work-Systemen nicht trivial \cite[S. 2]{cwo_pow}. 
Miner können ihre Hardware zum Minen benutzen, Kryptowährung, die zum \textit{Staken} benötigt wird, ist initial jedoch noch nicht verteilt. 
Auf dieses \textit{Bootstrapping Problem} wird in Abschnitt \ref{subsec:bootstrapping} näher eingegangen.

\subsection{Slots und Epochen}

Slots sind diskrete Einheiten von Zeit. Ein Block ist immer genau einem Slot zugeordnet. Eine feste Sequenz von Slots kann zu einer Epoche zusammengefasst werden. 
\cite[S. 3ff.]{ouroboros}

\subsection{Forks und Branches}

Forks und Branches der Blockchain sind analog zu Bäumen in der theoretischen Informatik definiert. Zur Verdeutlichen zeigt Abbildung \ref{fig:branches} ein Beispiel aus Git.


\begin{figure}[htb] 
	\centerline{\includegraphics*[width=0.8\textwidth]{img/basic-branching-6}}
\caption{Branches am Beispiel von Git\protect\footnotemark}
\label{fig:branches}
\end{figure}

\footnotetext{Scott Chacon, Ben Straub. Pro Git. Abbildung 20. \url{https://git-scm.com/book/en/v2/Git-Branching-Basic-Branching-and-Merging}, aufgerufen am 27. Feburuar 2018}

Zu sehen sind eine Verkettung von \textit{Commits}, diese sollen Blöcken entsprechen. Nach Block $C2$ gab es einen Fork. Es gibt nun zwei Branches die mit $C4$ und $C5$ enden.

Jedem Slot wird in der Regel ein Block zugeordnet, wenn keine Branches existieren. Die Ausnahmen bilden im Beispiel hier $C4$ und $C3$.

\subsection{Angriff}
\label{subsec:attack}

Ein Angriff ist der Versuch eine alternative Blockchain, bzw. einen alternativen Branch, bereitzustellen, die aufgrund von Metriken wie z.B. der \textit{Longest Chain} Regel, von den anderen Teilnehmern im Netzwerk akzeptiert wird.
Oder ein Versuch die Funktionsweise dieser Metriken zu stören.
