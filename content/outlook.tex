Der größte Vorteil von Proof-of-Work-Protokollen ist ihre Einfachheit. Durch das Mining sind negative Anreize implizit vorhanden und müssen nicht explizit ausgestaltet werden.

Mit steigendem Verständnis für Kryptowährungen kann diese Einfachheit jedoch einen Nachteil darstellen. Die Möglichkeit Anreize selbst frei zu gestalten ist verwehrt und somit kann das Verhalten der Partizipanten nicht  nach Belieben beeinflusst werden.

Das Proof-of-Work-Verfahren für Kryptowährungen ist trotzdem nicht obsolet. Es eignet sich als gute Lösung für die initiale Verteilung der Münzen -- das Bootstrapping Problem.
Und so ist eine Koexistenz dieser beiden Methoden vorstellbar und sogar wünschenswert.

Interessante Kryptowährungen im Kontext von Proof of Stake sind momentan Ethereum\footnote{Siehe \url{https://www.ethereum.org/}} und Cardano\footnote{Siehe \url{https://www.cardanohub.org/}}. 
Beide sind beliebte Top 10\footnote{Daten zur Markkaptitalisierung sind entnommen aus \url{https://coinmarketcap.com/}, aufgerufen am 1. März 2018} Münzen mit einer hohen Marktkapitalisierung.
Ethereum setzt aktuell auf Proof of Work und plant den graduellen Umstieg auf Proof of Stake\footnote{Siehe auch \url{https://cryptocanucks.com/metropolis-part-2-constantinople-ethereum/}}. Das Problem des Bootstrappings wurde hier also umgangen. Cardano setzt auf ein reines Proof-of-Stake-Verfahren und nutzte einen ICO um die Münzen initial zu verteilen. Es ist die Einführung von Stake Delegation geplant\footnote{Stake Delegation wird mit dem Shelley Release erscheinen, siehe \url{https://cardanoroadmap.com/\#shelley-decentralised}}. Dies gibt kleinen Usern die Möglichkeit sich zu Pools zusammenzuschließen und -- parallel zu Miningpools -- gemeinsam zu validierern.

Die Adaption durch größere Projekte wird Proof of Stake mehr Gewicht geben und das Interesse und die Forschung in dem Bereich vorantreiben.